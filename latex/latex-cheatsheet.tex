%%%%%%%%%%%%%%%%%%%%%%%%%%%%%%%%%%%%%%%%%%%%%%%%%%%%%%%%%%%%%%%%%%%%%%%%%%%%%
% Copyright (C) 2011 Michael Corral.
%
% Permission is granted to copy, distribute and/or modify this document
% under the terms of the GNU Free Documentation License,
% Version 1.3 or any later version published by the Free Software Foundation;
% with no Invariant Sections, no Front-Cover Texts, and no Back-Cover Texts.
% A copy of the license is included in the file gnufdl.txt, and is also
% available at http://www.gnu.org/licenses/fdl-1.3.html.
%%%%%%%%%%%%%%%%%%%%%%%%%%%%%%%%%%%%%%%%%%%%%%%%%%%%%%%%%%%%%%%%%%%%%%%%%%%%%
\documentclass[a4paper,12pt]{article}
\usepackage{vntex}
\usepackage[hmargin=0.5in,vmargin=0.5in,letterpaper,includefoot]{geometry}
\usepackage{xcolor,graphicx}
\usepackage{mathtools}
\usepackage{fancybox,fancyvrb}
\usepackage{textcomp}
\usepackage[unicode]{hyperref}
\hypersetup{pdftitle={LaTeX Cheatsheet},
            pdfsubject={LaTeX},
            pdfauthor={Trần Đình Anh Tuấn},
            pdfkeywords={cheatsheet},
            pdfdisplaydoctitle=true,pdfborder={0 0 0},
            colorlinks=true,urlcolor=blue}
%\usepackage{breakurl}
\usepackage{xspace}
\newcommand{\latex}{\LaTeX\xspace}
\begin{document}
\begin{center}
 \Large\textbf{\latex Cheatsheet}
\end{center}
Đây là một tài liệu dùng để tra cứu nhanh các lệnh thường dùng trong \latex

\section*{Cấu trúc tài liệu}
Cấu trúc cơ bản thường dùng trong mọi tài liệu \latex. Trong phần này bao gồm
một số lệnh dùng với tiếng Việt.

\begin{Verbatim}[frame=single]
%%% Phần khai báo (Preamble)
%%%

%sử dụng lớp tài liệu article với kiểu giấy A4, cỡ chữ 12pt
\documentclass[a4paper,12pt]{article}

\usepackage{vntex} %sử dụng gói vntex để soạn thảo tiếng Việt
\usepackage{amsmath,amssymb} %gõ kí hiệu, công thức Toán của AMS
\usepackage{xcolor,graphicx} %màu sắc, hình ảnh

%Tiêu chuẩn văn bản Việt Nam, tùy ý khai báo:
\usepackage[top=2cm,bottom=2cm,left=3cm,right=2cm]{geometry} %cỡ tài liệu
\usepackage{times} %font chữ kiểu Times New Roman
\usepackage{indentfirst} %thụt đầu dòng đoạn văn
\renewcommand{\baselinestretch}{1.7} %dãn cách dòng 1.5 như M$Word

%%% Phần nội dung
%%%
\begin{document}

\title{Tiêu đề}
\author{Tác giả}
\date{\today}
\maketitle

\section{Tiết đoạn đánh số}
\subsection{Tiết đoạn con}
\section*{Tiết đoạn không đánh số}

\end{document}
\end{Verbatim}

\section*{Định dạng kiểu chữ}

\begin{center}
 \begin{tabular}{|l|l|l|@{}}
\hline
\textbf{Kiểu chữ} & \textbf{Lệnh} & \textbf{Hiển thị}\\
\hline
Đậm & \texttt{\textbackslash textbf$\lbrace$chữ đậm$\rbrace$} & \textbf{chữ đậm}\\
Nghiêng & \texttt{\textbackslash emph$\lbrace$chữ nghiêng$\rbrace$} & \emph{chữ
nghiêng}\\
Đậm nghiêng & \texttt{\textbackslash textbf$\lbrace$\textbackslash
emph$\lbrace$chữ đậm nghiêng$\rbrace\rbrace$} & \textbf{\emph{chữ đậm nghiêng}}\\
Gạch dưới & \texttt{\textbackslash underline$\lbrace$chữ gạch dưới$\rbrace$} &
\underline{chữ gạch dưới}\\
In hoa & \texttt{\textbackslash textsc$\lbrace$chữ in hoa$\rbrace$} &
\textsc{chữ in hoa}\\
Giãn cách đơn (monospaced) & \texttt{\textbackslash texttt$\lbrace$chữ giãn cách
đơn$\rbrace$} & \texttt{chữ giãn cách đơn}\\
Chữ không chân (sans serif) & \texttt{\textbackslash textsf$\lbrace$chữ không
chân$\rbrace$} & \textsf{chữ không chân}\\
Kiểu số mũ & \texttt{chữ\textbackslash textsuperscript$\lbrace$số mũ$\rbrace$}
& chữ\textsuperscript{số mũ}\\
Cỡ tiny & \texttt{\textbackslash tiny$\lbrace$cỡ tiny$\rbrace$} & \tiny{cỡ tiny}\\
Cỡ superscript & \texttt{\textbackslash scriptsize$\lbrace$cỡ
superscript$\rbrace$} & \scriptsize{cỡ superscript}\\
Cỡ footnote & \texttt{\textbackslash footnotesize$\lbrace$cỡ footnote$\rbrace$}
& \footnotesize{cỡ footnote}\\
Cỡ small & \texttt{\textbackslash small$\lbrace$cỡ small$\rbrace$} & \small{cỡ small}\\
Cỡ normal & \texttt{\textbackslash normalsize$\lbrace$cỡ normal$\rbrace$} &
\normalsize{cỡ normal}\\
Cỡ large & \texttt{\textbackslash large$\lbrace$cỡ large$\rbrace$} & \large{cỡ large}\\
Cỡ larger & \texttt{\textbackslash Large$\lbrace$cỡ larger$\rbrace$} &
\Large{cỡ larger}\\
Cỡ largest & \texttt{\textbackslash LARGE$\lbrace$cỡ largest$\rbrace$} &
\LARGE{cỡ largest}\\
Cỡ huge & \texttt{\textbackslash huge$\lbrace$cỡ huge$\rbrace$} & \huge{cỡ huge}\\
Cỡ hugest & \texttt{\textbackslash Huge$\lbrace$cỡ hugest$\rbrace$} & \Huge{cỡ hugest}\\
Màu xanh\footnotemark & \texttt{\textbackslash
textcolor$\lbrace$blue$\rbrace\lbrace$màu xanh$\rbrace$} & \textcolor{blue}{màu
xanh}\\
\hline
\end{tabular}
\end{center}\footnotetext{Để sử dụng màu sắc phải nạp gói \texttt{xcolor} trong
phần khai báo}

\section*{Danh sách}

\begin{description}
    \item [môi trường \texttt{enumerate}] để tạo bảng danh sách đánh số kiểu
        Arabic.
    \item [môi trường \texttt{itemize}] để tạo danh sách kiểu chấm điểm.
    \item [môi trường \texttt{description}] để tạo danh sách mô tả.
\end{description}

\vspace{4mm}
\begin{minipage}[c]{3.5in}
\begin{Verbatim}[frame=single]
\begin{enumerate}
 \item Đánh số kiểu Arabic thứ nhất
  \begin{itemize}
   \item Đây là mục chấm điểm.
   \item Đây là mục chấm điểm
  \end{itemize}
 \item Đây là mục đánh số thứ hai.
\end{enumerate}
\end{Verbatim}
\end{minipage}
\begin{minipage}[c]{3.5in}
\begin{center}
\shadowbox{\parbox{3in}{%
\begin{enumerate}
 \item Đánh số kiểu Arabic thứ nhất
  \begin{itemize}
   \item Đây là mục chấm điểm.
   \item Đây là mục chấm điểm
  \end{itemize}
 \item Đây là mục đánh số thứ hai.
\end{enumerate}
}}
\end{center}
\end{minipage}\vspace{4mm}

\section*{Bảng}

Môi trường \texttt{tabular} dùng để tạo bảng, định dạng như sau:

\begin{Verbatim}[frame=single]
\begin{tabular}{kiểu gióng hàng(c,l hoặc r); mỗi chữ đại diện một cột}
    cột 1 hàng 1 & cột 2 hàng 1 & cột 3 hàng 1\\
    cột 1 hàng 2 & cột 2 hàng 2 & cột 3 hàng 3\\
\end{tabular}
\end{Verbatim}

Bên dưới là một ví dụ về bảng có ba cột hai hàng; cột thứ nhất gióng hàng giữa
\texttt{c}, cột thứ hai gióng hàng trái \texttt{l} và cột thứ ba gióng hàng
phải \texttt{r}; mỗi cột trong hàng phân cách nhau bằng kí tự \&, mỗi hàng phân
cách bằng hai dấu \textbackslash\textbackslash:

\vspace{4mm}
\begin{minipage}[c]{3.5in}
\begin{Verbatim}[frame=single]
\begin{tabular}{clr}
 Cột 1 & Cột 2 & Cột 3\\
 Đây & là & hàng\\
\end{tabular}
\end{Verbatim}
\end{minipage}
\begin{minipage}[c]{3.5in}
\begin{center}
\shadowbox{\parbox{3in}{%
\begin{tabular}{clr}
 Cột 1 & Cột 2 & Cột 3\\
 Đây & là & hàng\\
\end{tabular}
}}
\end{center}
\end{minipage}\vspace{4mm}

\noindent \textbf{Ghi chú}: Mặc định, \latex sẽ không kẻ hàng ngăn cách các cột
và hàng. Để kẻ hàng ngăn cách cột, ta thêm kí tự \texttt{|} vào phần gióng hàng
các cột như thế này \verb+\begin{tabular}{|c|l|r|}+. Để kẻ hàng ngăn cách hàng,
ta thêm \texttt{\textbackslash hline} vào giữa các hàng.

\section*{Đóng khung chữ, đoạn văn}

\begin{flushleft}
\begin{minipage}[c]{3.5in}
\begin{Verbatim}[frame=single]
Đóng khung \fbox{chữ} rất dễ.

\fbox{\parbox{\textwidth}{Đóng khung
đoạn văn cũng không khó. Đoạn văn được
đóng khung thể hiện tầm quan trọng.}}
\end{Verbatim}
\end{minipage}\quad
\begin{minipage}[c]{3.5in}
\begin{center}
\shadowbox{\parbox{3.65in}{%
Đóng khung \fbox{chữ} rất dễ.

\fbox{\parbox{\textwidth}{Đóng khung
đoạn văn cũng không khó. Đoạn văn được
đóng khung thể hiện tầm quan trọng.}}
}}
\end{center}
\end{minipage}
\end{flushleft}

\section*{Hình ảnh}

Chèn hình ảnh vào văn bản:
\begin{verbatim}
\includegraphics[thuộctính1=...,thuộctính2=...]{tên_tệp_ảnh}
\end{verbatim}

\begin{center}
\begin{tabular}{|l|l|}
    \hline
    \textbf{Thuộc tính = Giá trị} & \textbf{Chú thích}\\
    \hline
    width=xx & Chiều rộng ảnh = xx \emph{pt, mm, in}\\
    \hline
    height=xx & Chiều cao ảnh = xx \emph{pt, mm, in}\\
    \hline
    keepaspectratio=true|false & giữ nguyên tỉ lệ ảnh \emph{true} hoặc
    \emph{false}\\
    \hline
    scale=xx & phóng to, thu nhỏ tỉ lệ xx\\
    \hline
    angle=xx & xoay ảnh góc \emph{xx} độ\\
    \hline
\end{tabular}
\end{center}


\end{document}
